\documentclass[12pt]{article}         
\usepackage{fullpage}
\usepackage[shortlabels]{enumitem}
\usepackage{amsmath}
\usepackage{amsfonts}
\usepackage{graphicx}
\usepackage{listings}
\usepackage{xcolor}
\usepackage[colorlinks=true, urlcolor=blue, linkcolor=red]{hyperref}
\usepackage{subcaption}

\definecolor{codegreen}{rgb}{0,0.6,0}
\definecolor{codegray}{rgb}{0.5,0.5,0.5}
\definecolor{codepurple}{rgb}{0.58,0,0.82}
\definecolor{backcolour}{rgb}{0.95,0.95,0.92}

\lstdefinestyle{mystyle}{
    backgroundcolor=\color{backcolour},   
    commentstyle=\color{codegreen},
    keywordstyle=\color{magenta},
    numberstyle=\tiny\color{codegray},
    stringstyle=\color{codepurple},
    basicstyle=\ttfamily\footnotesize,
    breakatwhitespace=false,         
    breaklines=true,                 
    captionpos=b,                    
    keepspaces=true,                 
    numbers=left,                    
    numbersep=5pt,                  
    showspaces=false,                
    showstringspaces=false,
    showtabs=false,                  
    tabsize=2
}

\lstset{style=mystyle}

%\usepackage{amsmath}
%\usepackage{amssymb}
%\usepackage{enumitem}

\title{Twister Tuning: Optimizing Insurance Claim Handling To Minimize Costs}
\author{Karl Kreuze, Kevin Chang, Gabriel Maravilla, Brian Choi, Nicholas Nguyen}

\begin{document}
\maketitle

\section*{Abstract}
This document presents the results of a quantum algorithm designed to optimize insurance claim Handling
to minimize costs. A classification of tornado severities are collected and converted to a binary
encoding and then input to a quadratic unconstrained binary optimization (QUBO) problem. 
This conversion to a QUBO problem allows a quantum computer to efficiently solve the problem using
a superposition of states to solve the problem without needing to exhaustively search through all possible
solutions. With the increasing change in climate, the need for insurance companies to
efficiently handle claims is becoming more important. Our derivations show that quantum computers
will be very useful in the future for solving these types of problems.

\section*{Derivation of Cost Function}
We are given the following variables as given by a tornado severity classification:
\begin{align*}
    N_{ik} &= \text{$\#$ of tornadoes of severity $i$ in week $k$}\\
    N_{i} &= \sum_{k=1}^{K} N_{ik} = \text{$\#$ of tornadoes of severity $i$}\\
    M_{ij} &= \text{claims of severity $i$ servicable by skill $j$ per day}\\
    W_{i} &= \text{weeks where 90$\%$ of claims must be serviced}
\end{align*}
\newpage 
\noindent
We are additionally given the following constraints:
\begin{align*}
    \sum_{k=1}^{W_i} \sum^{j} 7X_{ijk}M_{ij} &\geq \min(\sum_{k=1}^{W_i}N_{ik}, 0.9N_{i}), \quad \forall i\\
    \sum_{k=1}^{4}\sum^{j} X_{ijk}M_{ij} &\geq N_{i}, \quad \forall i\\
    X_{ijk} &\geq 0, \quad \forall i,j,k\\
\end{align*}
where $X_{ijk}$ is the number of workers servicing claim of severity $i$ serviced by skill $j$ in week $k$.

The top constraint says that 90$\%$ of claims must be serviced in $W_i$ weeks.
The second constraint says that all the claims must be serviced in 4 weeks.

\subsection*{Designing the Quantum Register}
In the given spreadsheet, we have a thousand claims total, so we figured that 40 bits in a register,
with 8 bits for each level of severity, would be enough to represent the number of workers servicing claims of severity $i$.
Thus, we will put a 40 bit superposition state into the QUBO circuit.
If we get an output of \texttt{0b100000001}, we may split up the bits into 5 groups of 8 bits, and
convert each block into the corresponding skill level's number of workers. In this example,
we would have one worker of skill level 1, and 1 worker of skill level 2.

\subsection*{Sum of Workers}
Our cost function will start with the sum of workers of various skill levels.
\begin{align*}
    J(x)_{x \in \{0,1\}^{40}} &= \sum_{n=1}^{N} X_{n}\\
\end{align*}
Next, we'll add penalties corresponding to the constraints.
\begin{align*}
    J(x)_{x \in \{0,1\}^{40}} &= \sum_{n=1}^{N} X_{n} +
    \lambda_0\left(\sum^{j}\sum^{i}\left(X_{j}M_{ij} - N_{i}\right)\right)^2 + \lambda_1\left(\sum^{j} \text{relu}(-X_{j})\right)^2\\
\end{align*}
\newpage
\noindent
The first penalty grows whenever we are not servicing all the claims.
The second penalty grows whenever we have negative values in the register.
The relu function is defined as:
\begin{align*}
    \text{relu}(x) &= \begin{cases}
        0 & x < 0\\
        x & x \geq 0
    \end{cases}
\end{align*}
We'll make the penalty values very large, so that the cost function will be minimized when the constraints are satisfied.
\section*{Testing}
Unfortunately, we were unable to test the QUBO circuit because of the release of Qisket 2 causing
dependency issues with Nexus Labs
\section*{Additional constraints}
We'd need to match workers to weeks to work, and match them to the claims they are servicing.
This new cost function would include our given constraint above about servicing 90$\%$ of claims in $W_i$ weeks.
\section*{Future Work}
Additional constraints such as commute time, geography, and virtual versus in-person work would
need their own cost functions to be evaluated. For example, a penalty could be added for a virtual
worker servicing an in-person claim, and this could be represented by a function that checks the
skill of the worker and compares it to the severity of the claim, and adds a penalty based on the given table.


\end{document}